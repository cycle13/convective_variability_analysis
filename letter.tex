% \documentclass[a4paper, 12pt, draft]{article}
\documentclass[a4paper, 12pt]{article}
\usepackage{natbib}
\usepackage{graphicx}
\usepackage{tikz}   % For the for loop
\usepackage[top=3cm, bottom=3cm, left=2.5cm, right=2.5cm]{geometry}
\graphicspath{{/home/s/S.Rasp/Dropbox/figures/PhD/variance/}}

\title{Convective variability in summertime mid-latitude weather}

\begin{document}

% \section*{Key Points}
% \begin{itemize}
%  \item 
% \end{itemize}
% 

\section*{Abstract}


\section{Introduction}
\begin{itemize}
% Importance of convective variability
 \item Representing variability of convection through stochastic parameterizations is important for $\Delta x$ between 100 and 1\,km to improve reliability and reduce biases \citep{Jones2011, Berner2016}.
% CC06 plus applications and benefits
 \item \cite{Craig2006} proposed a quantitative model of convective variability based on statistical mechanics assumptions. 
\end{itemize}
\begin{equation} \label{eq:m_dist}
 P(m) = \frac{1}{\langle m \rangle}e^{-m/\langle m \rangle}
\end{equation}
and 
\begin{equation} \label{eq:N_dist}
 P(N) = \frac{\langle N \rangle^N}{N!}e^{-\langle N \rangle}.
\end{equation}
Combing these, the distribution of the total mass flux $M$ is given by
\begin{equation} \label{eq:M_dist}
 P(M) = \left( \frac{\langle N \rangle}{\langle m \rangle} \right)^{1/2} e^{-\langle N \rangle} M^{-1/2} e^{-M/\langle m \rangle} I_1\left[ 2 \left( \frac{\langle N \rangle}{\langle m \rangle} M \right)^{1/2} \right],
\end{equation}
where $I_1(x)$is the modified Bessel function of order 1. For large (small) values of $\langle N \rangle$ the shape of this function resembles a Gaussian (Poisson) distribution. The normalized variance is given by 
\begin{equation} \label{eq:M_var}
 \frac{\langle (\delta M)^2 \rangle}{\langle M \rangle^2} = \frac{2}{\langle N \rangle}.
\end{equation}
Note that $\langle M \rangle = \langle N \rangle \langle m \rangle$. Equivalently,
\begin{equation} \label{eq:M_std}
 \langle (\delta M)^2 \rangle^{1/2}= \sqrt{2 \langle M \rangle \langle m \rangle}.
\end{equation}

\begin{itemize}
% Benefits of stoachastic
\item The \cite{Plant2008} stochastic parameterization is based on the CC06 theory. Its positive impact has been shown by several studies \citep{Kober2015, Selz2015b, Wang2016}.
% Davoudi and Davies, impact of organization in idealized
\item Studies by \cite{Davoudi2010} and \cite{Davies2008} have shown deviations of the theory in simulations with a diurnal cycle caused by organization of clouds. 
\end{itemize}

% My RQs
\begin{itemize}
 \item Primary research question: Does CC06 variance scaling hold up in simulations of real weather?
 \item Secondary research question: What causes systematic deviations?
\end{itemize}


\section{Simulation Setup}
% Ensemble setup and period
Simulations were performed for a continuous 12 day period from 28 May to 8 June 2016 over Germany. This period was characterized by heavy convective precipitation, which was associated with several tornado and flooding events. The model used was the COSMO model \citep{Baldauf2011} with a horizontal grid spacing of 2.8\,km. The settings were taken from the operational COSMO-DE setup, with one exception, the stochastic boundary layer scheme which will be described below. Deep convection is treated explicitly. The domain size is 357 grid points in either direction with the domain centered at 10E/50N. For the analysis a 256 by 256 grid point domain (roughly 717\,km) was chosen to allow room for boundary spin-up. The aim of the simulation setup was to create an ensemble of simulations which have sufficiently similar large scale conditions, but displaced convection. This allows then to treat each member representing a different realization of convection associated with the same mean forcing. A 50 member ensemble was initialized at 00UTC for each of the 12 days, and was run for 24\,h. Initial and boundary conditions were the same for all members and were interpolated from the deterministic COSMO-EU analysis and forecast ($\Delta x =$ 7\,km). 

% KC16
Each of the members was perturbed using a stochastic boundary layer perturbation scheme \citep{Kober2016}. This scheme adds perturbations to the model tendency equations based on sub-grid variances computed in the boundary layer scheme. 

% Are simulations valid
To test the realism of our simulations 

% Computation of M stuff


\section{Results}
% Std vs mean

% dependence on diurnal cycle and clustering


\section{Discussion and Conclustions}


\section{Supplemental Material}


 
\end{document}
